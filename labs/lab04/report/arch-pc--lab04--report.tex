% Options for packages loaded elsewhere
% Options for packages loaded elsewhere
\PassOptionsToPackage{unicode}{hyperref}
\PassOptionsToPackage{hyphens}{url}
%
\documentclass[
  english,
  russian,
  12pt,
  a4paper,
  DIV=11,
  numbers=noendperiod]{scrreprt}
\usepackage{xcolor}
\usepackage{amsmath,amssymb}
\setcounter{secnumdepth}{5}
\usepackage{iftex}
\ifPDFTeX
  \usepackage[T1]{fontenc}
  \usepackage[utf8]{inputenc}
  \usepackage{textcomp} % provide euro and other symbols
\else % if luatex or xetex
  \usepackage{unicode-math} % this also loads fontspec
  \defaultfontfeatures{Scale=MatchLowercase}
  \defaultfontfeatures[\rmfamily]{Ligatures=TeX,Scale=1}
\fi
\usepackage{lmodern}
\ifPDFTeX\else
  % xetex/luatex font selection
\fi
% Use upquote if available, for straight quotes in verbatim environments
\IfFileExists{upquote.sty}{\usepackage{upquote}}{}
\IfFileExists{microtype.sty}{% use microtype if available
  \usepackage[]{microtype}
  \UseMicrotypeSet[protrusion]{basicmath} % disable protrusion for tt fonts
}{}
\usepackage{setspace}
% Make \paragraph and \subparagraph free-standing
\makeatletter
\ifx\paragraph\undefined\else
  \let\oldparagraph\paragraph
  \renewcommand{\paragraph}{
    \@ifstar
      \xxxParagraphStar
      \xxxParagraphNoStar
  }
  \newcommand{\xxxParagraphStar}[1]{\oldparagraph*{#1}\mbox{}}
  \newcommand{\xxxParagraphNoStar}[1]{\oldparagraph{#1}\mbox{}}
\fi
\ifx\subparagraph\undefined\else
  \let\oldsubparagraph\subparagraph
  \renewcommand{\subparagraph}{
    \@ifstar
      \xxxSubParagraphStar
      \xxxSubParagraphNoStar
  }
  \newcommand{\xxxSubParagraphStar}[1]{\oldsubparagraph*{#1}\mbox{}}
  \newcommand{\xxxSubParagraphNoStar}[1]{\oldsubparagraph{#1}\mbox{}}
\fi
\makeatother


\usepackage{longtable,booktabs,array}
\usepackage{calc} % for calculating minipage widths
% Correct order of tables after \paragraph or \subparagraph
\usepackage{etoolbox}
\makeatletter
\patchcmd\longtable{\par}{\if@noskipsec\mbox{}\fi\par}{}{}
\makeatother
% Allow footnotes in longtable head/foot
\IfFileExists{footnotehyper.sty}{\usepackage{footnotehyper}}{\usepackage{footnote}}
\makesavenoteenv{longtable}
\usepackage{graphicx}
\makeatletter
\newsavebox\pandoc@box
\newcommand*\pandocbounded[1]{% scales image to fit in text height/width
  \sbox\pandoc@box{#1}%
  \Gscale@div\@tempa{\textheight}{\dimexpr\ht\pandoc@box+\dp\pandoc@box\relax}%
  \Gscale@div\@tempb{\linewidth}{\wd\pandoc@box}%
  \ifdim\@tempb\p@<\@tempa\p@\let\@tempa\@tempb\fi% select the smaller of both
  \ifdim\@tempa\p@<\p@\scalebox{\@tempa}{\usebox\pandoc@box}%
  \else\usebox{\pandoc@box}%
  \fi%
}
% Set default figure placement to htbp
\def\fps@figure{htbp}
\makeatother



\ifLuaTeX
\usepackage[bidi=basic,provide=*]{babel}
\else
\usepackage[bidi=default,provide=*]{babel}
\fi
% get rid of language-specific shorthands (see #6817):
\let\LanguageShortHands\languageshorthands
\def\languageshorthands#1{}


\setlength{\emergencystretch}{3em} % prevent overfull lines

\providecommand{\tightlist}{%
  \setlength{\itemsep}{0pt}\setlength{\parskip}{0pt}}



 
\usepackage[backend=biber,langhook=extras,autolang=other*]{biblatex}
\addbibresource{bib/cite.bib}

\usepackage[]{csquotes}

\usepackage{indentfirst}
\usepackage{float}
\floatplacement{figure}{H}
\IfFileExists{plex-otf.sty}{
  %% Full TeXlive
  \usepackage[math,RM={Scale=0.94},SS={Scale=0.94},SScon={Scale=0.94},TT={Scale=MatchLowercase,FakeStretch=0.9},DefaultFeatures={Ligatures=Common}]{plex-otf}
}{
  %% TinyTeX
  \usepackage{libertine}
}
\KOMAoption{captions}{tableheading}
\makeatletter
\@ifpackageloaded{caption}{}{\usepackage{caption}}
\AtBeginDocument{%
\ifdefined\contentsname
  \renewcommand*\contentsname{Содержание}
\else
  \newcommand\contentsname{Содержание}
\fi
\ifdefined\listfigurename
  \renewcommand*\listfigurename{Список иллюстраций}
\else
  \newcommand\listfigurename{Список иллюстраций}
\fi
\ifdefined\listtablename
  \renewcommand*\listtablename{Список таблиц}
\else
  \newcommand\listtablename{Список таблиц}
\fi
\ifdefined\figurename
  \renewcommand*\figurename{Рисунок}
\else
  \newcommand\figurename{Рисунок}
\fi
\ifdefined\tablename
  \renewcommand*\tablename{Таблица}
\else
  \newcommand\tablename{Таблица}
\fi
}
\@ifpackageloaded{float}{}{\usepackage{float}}
\floatstyle{ruled}
\@ifundefined{c@chapter}{\newfloat{codelisting}{h}{lop}}{\newfloat{codelisting}{h}{lop}[chapter]}
\floatname{codelisting}{Список}
\newcommand*\listoflistings{\listof{codelisting}{Листинги}}
\makeatother
\makeatletter
\makeatother
\makeatletter
\@ifpackageloaded{caption}{}{\usepackage{caption}}
\@ifpackageloaded{subcaption}{}{\usepackage{subcaption}}
\makeatother
\usepackage{bookmark}
\IfFileExists{xurl.sty}{\usepackage{xurl}}{} % add URL line breaks if available
\urlstyle{same}
\hypersetup{
  pdftitle={ОТЧЕТ ПО ЛАБОРАТОРНОЙ РАБОТЕ № 4},
  pdfauthor={Сидорова А.А.},
  pdflang={ru-RU},
  hidelinks,
  pdfcreator={LaTeX via pandoc}}


\title{\textbf{ОТЧЕТ ПО ЛАБОРАТОРНОЙ РАБОТЕ № 4}}
\usepackage{etoolbox}
\makeatletter
\providecommand{\subtitle}[1]{% add subtitle to \maketitle
  \apptocmd{\@title}{\par {\large #1 \par}}{}{}
}
\makeatother
\subtitle{\emph{дисциплина: Архитектура компьютера}}
\author{Сидорова А.А.}
\date{}
\begin{document}
\maketitle

\renewcommand*\contentsname{Содержание}
{
\setcounter{tocdepth}{1}
\tableofcontents
}
\listoffigures
\listoftables

\setstretch{1.5}
\chapter{Цель
работы}\label{ux446ux435ux43bux44c-ux440ux430ux431ux43eux442ux44b}

Освоение процедуры кампиляции и сборки программ, написанных на
ассемблере NASM.

\chapter{Задание для самостоятельной
работы}\label{ux437ux430ux434ux430ux43dux438ux435-ux434ux43bux44f-ux441ux430ux43cux43eux441ux442ux43eux44fux442ux435ux43bux44cux43dux43eux439-ux440ux430ux431ux43eux442ux44b}

\begin{enumerate}
\def\labelenumi{\arabic{enumi}.}
\tightlist
\item
  В каталоге \textasciitilde/work/arch-pc/lab04 с помощью команды cp
  создайте копию файла hello.asm с именем lab4.asm
\item
  С помощью любого текстового редактора внесите изменения в текст
  программы в файле lab4.asm так, чтобы вместо Hello world! на экран
  выводилась строка с вашими фамилией и именем.
\item
  Оттранслируйте полученный текст программы lab4.asm в объектный файл.
  Выполните компоновку объектного файла и запустите получившийся
  исполняемый файл.
\item
  Скопируйте файлы hello.asm и lab4.asm в Ваш локальный репозиторий в
  ката- лог \textasciitilde/work/study/2023-2024/\enquote{Архитектура
  компьютера}/arch-pc/labs/lab04/. Загрузите файлы на Github.
\end{enumerate}

\chapter{Теоретическое
введение}\label{ux442ux435ux43eux440ux435ux442ux438ux447ux435ux441ux43aux43eux435-ux432ux432ux435ux434ux435ux43dux438ux435}

\section{Основные принципы работы
компьютера}\label{ux43eux441ux43dux43eux432ux43dux44bux435-ux43fux440ux438ux43dux446ux438ux43fux44b-ux440ux430ux431ux43eux442ux44b-ux43aux43eux43cux43fux44cux44eux442ux435ux440ux430}

Основными функциональными элементами любой электронно-вычислительной
машины (ЭВМ) являются центральный процессор, память и периферийные
устройства (рис. 4.1). Взаимодействие этих устройств осуществляется
через общую шину, к которой они подклю- чены. Физически шина
представляет собой большое количество проводников, соединяющих
устройства друг с другом. В современных компьютерах проводники выполнены
в виде элек- тропроводящих дорожек на материнской (системной) плате.
Основной задачей процессора является обработка информации, а также
организация координации всех узлов компьютера. В состав центрального
процессора (ЦП) входят следующие устройства: • арифметико-логическое
устройство (АЛУ) --- выполняет логические и арифметиче- ские действия,
необходимые для обработки информации, хранящейся в памяти; • устройство
управления (УУ) --- обеспечивает управление и контроль всех устройств
компьютера; • регистры --- сверхбыстрая оперативная память небольшого
объёма, входящая в со- став процессора, для временного хранения
промежуточных результатов выполнения инструкций; регистры процессора
делятся на два типа: регистры общего назначения и специальные регистры.
\#\# Ассемблер и язык ассемблера

Язык ассемблера (assembly language, сокращённо asm) ---
машинно-ориентированный язык низкого уровня. Можно считать, что он
больше любых других языков приближен к архитектуре ЭВМ и её аппаратным
возможностям, что позволяет получить к ним более полный доступ, нежели в
языках высокого уровня, таких как C/C++, Perl, Python и пр. NASM --- это
открытый проект ассемблера, версии которого доступны под различные
операционные системы и который позволяет получать объектные файлы для
этих систем. В NASM используется Intel-синтаксис и поддерживаются
инструкции x86-64.

\section{Процесс создания и обработки программы на языке
ассемблера}\label{ux43fux440ux43eux446ux435ux441ux441-ux441ux43eux437ux434ux430ux43dux438ux44f-ux438-ux43eux431ux440ux430ux431ux43eux442ux43aux438-ux43fux440ux43eux433ux440ux430ux43cux43cux44b-ux43dux430-ux44fux437ux44bux43aux435-ux430ux441ux441ux435ux43cux431ux43bux435ux440ux430}

Процесс создания ассемблерной программы можно изобразить в виде
следующей схемы (рис.0.1)

\begin{figure}[H]

{\centering \pandocbounded{\includegraphics[keepaspectratio]{home/aasidorova1/work/study/2025-2026/Архитектура компьютера/arch-pc/labs/lab04/report/image/Pamp4/screen0.1}}

}

\caption{рис.0.1}

\end{figure}%

\chapter{Выполнение лабораторной
работы}\label{ux432ux44bux43fux43eux43bux43dux435ux43dux438ux435-ux43bux430ux431ux43eux440ux430ux442ux43eux440ux43dux43eux439-ux440ux430ux431ux43eux442ux44b}

\section{1. Программа Hello
world!}\label{ux43fux440ux43eux433ux440ux430ux43cux43cux430-hello-world}

Создаю каталог для работы с программами на языке ассемблер NASM. (рис.1)

\begin{figure}[H]

{\centering \pandocbounded{\includegraphics[keepaspectratio]{./home/aasidorova1/work/study/2025-2026/Архитектура компьютера/arch-pc/labs/lab04/report/image/Pamp4/screen1.pdf}}

}

\caption{рис.1}

\end{figure}%

Перехожу в созданный каталог (рис.2)

\begin{figure}[H]

{\centering \pandocbounded{\includegraphics[keepaspectratio]{./home/aasidorova1/work/study/2025-2026/Архитектура компьютера/arch-pc/labs/lab04/report/image/Pamp4/screen2.pdf}}

}

\caption{рис.2}

\end{figure}%

Создаю текстовый файл с именем hello.asm и открываю его с помощью
текстового редактора gedit (рис.3-4)

\begin{figure}[H]

{\centering \pandocbounded{\includegraphics[keepaspectratio]{./home/aasidorova1/work/study/2025-2026/Архитектура компьютера/arch-pc/labs/lab04/report/image/Pamp4/screen3.pdf}}

}

\caption{рис.3}

\end{figure}%

\begin{figure}[H]

{\centering \pandocbounded{\includegraphics[keepaspectratio]{./home/aasidorova1/Pictures/Screenshots/Pamp4/screen4.png}}

}

\caption{рис.4}

\end{figure}%

ввела текст в этот файл (рис.5)

\begin{figure}[H]

{\centering \pandocbounded{\includegraphics[keepaspectratio]{./home/aasidorova1/work/study/2025-2026/Архитектура компьютера/arch-pc/labs/lab04/report/image/Pamp4/screen5.pdf}}

}

\caption{рис.5}

\end{figure}%

\section{2. Транслятор
NASM}\label{ux442ux440ux430ux43dux441ux43bux44fux442ux43eux440-nasm}

Компилирую текст программы «Hello World» и проверяю, что записался
объектный код в файл hello.o с помощью команды ls. (рис.6)

\begin{figure}[H]

{\centering \pandocbounded{\includegraphics[keepaspectratio]{./home/aasidorova1/work/study/2025-2026/Архитектура компьютера/arch-pc/labs/lab04/report/image/Pamp4/screen6.pdf}}

}

\caption{рис.6}

\end{figure}%

\section{3. Расширенный синтаксис командной строки
NASM}\label{ux440ux430ux441ux448ux438ux440ux435ux43dux43dux44bux439-ux441ux438ux43dux442ux430ux43aux441ux438ux441-ux43aux43eux43cux430ux43dux434ux43dux43eux439-ux441ux442ux440ux43eux43aux438-nasm}

Выполняю команду nasm -o obj.o -f elf -g -l list.lst hello.asm и с
помощью команды ls проверила работу команды.(рис.7)

\begin{figure}[H]

{\centering \pandocbounded{\includegraphics[keepaspectratio]{./home/aasidorova1/work/study/2025-2026/Архитектура компьютера/arch-pc/labs/lab04/report/image/Pamp4/screen7.pdf}}

}

\caption{рис.7}

\end{figure}%

\begin{figure}[H]

{\centering \pandocbounded{\includegraphics[keepaspectratio]{./home/aasidorova1/work/study/2025-2026/Архитектура компьютера/arch-pc/labs/lab04/report/image/Pamp4/screen8.pdf}}

}

\caption{рис.8}

\end{figure}%

\section{4. Компоновщик
LD}\label{ux43aux43eux43cux43fux43eux43dux43eux432ux449ux438ux43a-ld}

Передаю файл на обработку компановщику и проверяем исполнение команды, а
также создание файла hello. (рис.9)

\begin{figure}[H]

{\centering \pandocbounded{\includegraphics[keepaspectratio]{./home/aasidorova1/work/study/2025-2026/Архитектура компьютера/arch-pc/labs/lab04/report/image/Pamp4/screen9.pdf}}

}

\caption{рис.9}

\end{figure}%

Выполняю следующую команду и проверяю что она выполнилась с помощбю
команды ls. (рис.10)

\begin{figure}[H]

{\centering \pandocbounded{\includegraphics[keepaspectratio]{./home/aasidorova1/work/study/2025-2026/Архитектура компьютера/arch-pc/labs/lab04/report/image/Pamp4/screen10.pdf}}

}

\caption{рис.10}

\end{figure}%

\section{5.Запуск исполняемого
файла}\label{ux437ux430ux43fux443ux441ux43a-ux438ux441ux43fux43eux43bux43dux44fux435ux43cux43eux433ux43e-ux444ux430ux439ux43bux430}

Запускаю на выполнение созданный исполняемый файл, находящийся в текущем
каталоге, набраю в командной строке ./hello (рис.11)

\begin{figure}[H]

{\centering \pandocbounded{\includegraphics[keepaspectratio]{./home/aasidorova1/work/study/2025-2026/Архитектура компьютера/arch-pc/labs/lab04/report/image/Pamp4/screen11.pdf}}

}

\caption{рис.11}

\end{figure}%

\section{Задание для самостоятельной
работы}\label{ux437ux430ux434ux430ux43dux438ux435-ux434ux43bux44f-ux441ux430ux43cux43eux441ux442ux43eux44fux442ux435ux43bux44cux43dux43eux439-ux440ux430ux431ux43eux442ux44b-1}

\begin{enumerate}
\def\labelenumi{\arabic{enumi}.}
\tightlist
\item
  Создала копию файла hello.asm с именем lab4.asm. (рис.12)
\end{enumerate}

\begin{figure}[H]

{\centering \pandocbounded{\includegraphics[keepaspectratio]{./home/aasidorova1/work/study/2025-2026/Архитектура компьютера/arch-pc/labs/lab04/report/image/Pamp4/screen12.pdf}}

}

\caption{рис.12}

\end{figure}%

\begin{enumerate}
\def\labelenumi{\arabic{enumi}.}
\setcounter{enumi}{1}
\tightlist
\item
  Внесла изменение текст программы. Заменила Hello world! на моё имя и
  фамилия. Оттранслировала полученный текст программы lab4.asm в
  объектный файл. Выполнила компоновку объектного файла и запустила
  получившийся исполняемый файл.(рис.13)
\end{enumerate}

\begin{figure}[H]

{\centering \pandocbounded{\includegraphics[keepaspectratio]{./home/aasidorova1/work/study/2025-2026/Архитектура компьютера/arch-pc/labs/lab04/report/image/Pamp4/screen13.pdf}}

}

\caption{рис.13}

\end{figure}%

Скопировала файлы hello.asm и lab4.asm в Мой локальный репозиторий в
ката- лог \textasciitilde/work/study/2023-2024/\enquote{Архитектура
компьютера}/arch-pc/labs/lab04/. И загрузила файлы на Github.(рис.14)

\begin{figure}[H]

{\centering \pandocbounded{\includegraphics[keepaspectratio]{./home/aasidorova1/work/study/2025-2026/Архитектура компьютера/arch-pc/labs/lab04/report/image/Pamp4/screen14.pdf}}

}

\caption{рис.14}

\end{figure}%

\chapter{Вывод}\label{ux432ux44bux432ux43eux434}

В ходе этой лабораторной работы я приобрела практические навыки по
пользованию ассемблера NASM.


\printbibliography



\end{document}
