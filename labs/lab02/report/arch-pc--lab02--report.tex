% Options for packages loaded elsewhere
% Options for packages loaded elsewhere
\PassOptionsToPackage{unicode}{hyperref}
\PassOptionsToPackage{hyphens}{url}
%
\documentclass[
  english,
  russian,
  12pt,
  a4paper,
  DIV=11,
  numbers=noendperiod]{scrreprt}
\usepackage{xcolor}
\usepackage{amsmath,amssymb}
\setcounter{secnumdepth}{5}
\usepackage{iftex}
\ifPDFTeX
  \usepackage[T1]{fontenc}
  \usepackage[utf8]{inputenc}
  \usepackage{textcomp} % provide euro and other symbols
\else % if luatex or xetex
  \usepackage{unicode-math} % this also loads fontspec
  \defaultfontfeatures{Scale=MatchLowercase}
  \defaultfontfeatures[\rmfamily]{Ligatures=TeX,Scale=1}
\fi
\usepackage{lmodern}
\ifPDFTeX\else
  % xetex/luatex font selection
\fi
% Use upquote if available, for straight quotes in verbatim environments
\IfFileExists{upquote.sty}{\usepackage{upquote}}{}
\IfFileExists{microtype.sty}{% use microtype if available
  \usepackage[]{microtype}
  \UseMicrotypeSet[protrusion]{basicmath} % disable protrusion for tt fonts
}{}
\usepackage{setspace}
% Make \paragraph and \subparagraph free-standing
\makeatletter
\ifx\paragraph\undefined\else
  \let\oldparagraph\paragraph
  \renewcommand{\paragraph}{
    \@ifstar
      \xxxParagraphStar
      \xxxParagraphNoStar
  }
  \newcommand{\xxxParagraphStar}[1]{\oldparagraph*{#1}\mbox{}}
  \newcommand{\xxxParagraphNoStar}[1]{\oldparagraph{#1}\mbox{}}
\fi
\ifx\subparagraph\undefined\else
  \let\oldsubparagraph\subparagraph
  \renewcommand{\subparagraph}{
    \@ifstar
      \xxxSubParagraphStar
      \xxxSubParagraphNoStar
  }
  \newcommand{\xxxSubParagraphStar}[1]{\oldsubparagraph*{#1}\mbox{}}
  \newcommand{\xxxSubParagraphNoStar}[1]{\oldsubparagraph{#1}\mbox{}}
\fi
\makeatother


\usepackage{longtable,booktabs,array}
\usepackage{calc} % for calculating minipage widths
% Correct order of tables after \paragraph or \subparagraph
\usepackage{etoolbox}
\makeatletter
\patchcmd\longtable{\par}{\if@noskipsec\mbox{}\fi\par}{}{}
\makeatother
% Allow footnotes in longtable head/foot
\IfFileExists{footnotehyper.sty}{\usepackage{footnotehyper}}{\usepackage{footnote}}
\makesavenoteenv{longtable}
\usepackage{graphicx}
\makeatletter
\newsavebox\pandoc@box
\newcommand*\pandocbounded[1]{% scales image to fit in text height/width
  \sbox\pandoc@box{#1}%
  \Gscale@div\@tempa{\textheight}{\dimexpr\ht\pandoc@box+\dp\pandoc@box\relax}%
  \Gscale@div\@tempb{\linewidth}{\wd\pandoc@box}%
  \ifdim\@tempb\p@<\@tempa\p@\let\@tempa\@tempb\fi% select the smaller of both
  \ifdim\@tempa\p@<\p@\scalebox{\@tempa}{\usebox\pandoc@box}%
  \else\usebox{\pandoc@box}%
  \fi%
}
% Set default figure placement to htbp
\def\fps@figure{htbp}
\makeatother



\ifLuaTeX
\usepackage[bidi=basic,provide=*]{babel}
\else
\usepackage[bidi=default,provide=*]{babel}
\fi
% get rid of language-specific shorthands (see #6817):
\let\LanguageShortHands\languageshorthands
\def\languageshorthands#1{}


\setlength{\emergencystretch}{3em} % prevent overfull lines

\providecommand{\tightlist}{%
  \setlength{\itemsep}{0pt}\setlength{\parskip}{0pt}}



 
\usepackage[backend=biber,langhook=extras,autolang=other*]{biblatex}
\addbibresource{bib/cite.bib}

\usepackage[]{csquotes}

\usepackage{indentfirst}
\usepackage{float}
\floatplacement{figure}{H}
\IfFileExists{plex-otf.sty}{
  %% Full TeXlive
  \usepackage[math,RM={Scale=0.94},SS={Scale=0.94},SScon={Scale=0.94},TT={Scale=MatchLowercase,FakeStretch=0.9},DefaultFeatures={Ligatures=Common}]{plex-otf}
}{
  %% TinyTeX
  \usepackage{libertine}
}
\KOMAoption{captions}{tableheading}
\makeatletter
\@ifpackageloaded{caption}{}{\usepackage{caption}}
\AtBeginDocument{%
\ifdefined\contentsname
  \renewcommand*\contentsname{Содержание}
\else
  \newcommand\contentsname{Содержание}
\fi
\ifdefined\listfigurename
  \renewcommand*\listfigurename{Список иллюстраций}
\else
  \newcommand\listfigurename{Список иллюстраций}
\fi
\ifdefined\listtablename
  \renewcommand*\listtablename{Список таблиц}
\else
  \newcommand\listtablename{Список таблиц}
\fi
\ifdefined\figurename
  \renewcommand*\figurename{Рисунок}
\else
  \newcommand\figurename{Рисунок}
\fi
\ifdefined\tablename
  \renewcommand*\tablename{Таблица}
\else
  \newcommand\tablename{Таблица}
\fi
}
\@ifpackageloaded{float}{}{\usepackage{float}}
\floatstyle{ruled}
\@ifundefined{c@chapter}{\newfloat{codelisting}{h}{lop}}{\newfloat{codelisting}{h}{lop}[chapter]}
\floatname{codelisting}{Список}
\newcommand*\listoflistings{\listof{codelisting}{Листинги}}
\makeatother
\makeatletter
\makeatother
\makeatletter
\@ifpackageloaded{caption}{}{\usepackage{caption}}
\@ifpackageloaded{subcaption}{}{\usepackage{subcaption}}
\makeatother
\usepackage{bookmark}
\IfFileExists{xurl.sty}{\usepackage{xurl}}{} % add URL line breaks if available
\urlstyle{same}
\hypersetup{
  pdftitle={ОТЧЕТ ПО ЛАБОРАТОРНОЙ РАБОТЕ № 2},
  pdfauthor={Сидорова А.А.},
  pdflang={ru-RU},
  hidelinks,
  pdfcreator={LaTeX via pandoc}}


\title{ОТЧЕТ ПО ЛАБОРАТОРНОЙ РАБОТЕ № 2}
\usepackage{etoolbox}
\makeatletter
\providecommand{\subtitle}[1]{% add subtitle to \maketitle
  \apptocmd{\@title}{\par {\large #1 \par}}{}{}
}
\makeatother
\subtitle{дисциплина: Архитектура компьютера}
\author{Сидорова А.А.}
\date{}
\begin{document}
\maketitle

\renewcommand*\contentsname{Содержание}
{
\setcounter{tocdepth}{1}
\tableofcontents
}
\listoffigures
\listoftables

\setstretch{1.5}
\chapter{Цель
работы}\label{ux446ux435ux43bux44c-ux440ux430ux431ux43eux442ux44b}

Целью работы является изучение идеологии и применения средств контроля
версий, приобретение практических навыков по работе с системой контроля
версий git

\chapter{Задание для самостоятельной
работы}\label{ux437ux430ux434ux430ux43dux438ux435-ux434ux43bux44f-ux441ux430ux43cux43eux441ux442ux43eux44fux442ux435ux43bux44cux43dux43eux439-ux440ux430ux431ux43eux442ux44b}

\begin{enumerate}
\def\labelenumi{\arabic{enumi}.}
\tightlist
\item
  Создайте отчет по выполнению лабораторной работы в соответствующем ка
  талоге рабочего пространства (labs/lab02/report).
\item
  Скопируйте отчеты по выполнению предыдущих лабораторных работ в соо
  тветствующие каталоги созданного рабочего пространства.
\item
  Загрузите файлы на github.
\end{enumerate}

\chapter{Теоретическое
введение}\label{ux442ux435ux43eux440ux435ux442ux438ux447ux435ux441ux43aux43eux435-ux432ux432ux435ux434ux435ux43dux438ux435}

\section{Основные команды
git.}\label{ux43eux441ux43dux43eux432ux43dux44bux435-ux43aux43eux43cux430ux43dux434ux44b-git.}

\begin{longtable}[]{@{}
  >{\raggedright\arraybackslash}p{(\linewidth - 2\tabcolsep) * \real{0.1895}}
  >{\raggedright\arraybackslash}p{(\linewidth - 2\tabcolsep) * \real{0.8105}}@{}}
\toprule\noalign{}
\begin{minipage}[b]{\linewidth}\raggedright
Команда
\end{minipage} & \begin{minipage}[b]{\linewidth}\raggedright
Описание
\end{minipage} \\
\midrule\noalign{}
\endhead
\bottomrule\noalign{}
\endlastfoot
\texttt{git\ init} & создание основного дерева репозитория \\
\texttt{git\ pull} & получение обновлений (изменений) текущего дерева из
центрального репозитория \\
\texttt{git\ push\ \ \ \ \ \ \ \ \ \ \ \ \ \ \ \ \ \ \ \textbar{}\ отправка\ всех\ произведённых\ изменений\ локального\ дерева\ в\ центральный\ репозиторий\textbar{}\ \textbar{}}git
status\texttt{\textbar{}\ просмотр\ списка\ изменённых\ файлов\ в\ текущей\ директории\textbar{}\ \textbar{}}git
diff\texttt{\textbar{}\ просмотр\ текущих\ изменений\textbar{}\ \textbar{}}git
add
.\texttt{\textbar{}\ добавить\ все\ изменённые\ и/или\ созданные\ файлы\ и/или\ каталоги\textbar{}\ \textbar{}}git
add
имена\_файлов\texttt{\textbar{}\ добавить\ конкретные\ изменённые\ и/или\ созданные\ файлы\ и/или\ каталоги\textbar{}\ \ \ \ \ \ \ \ \ \ \ \ \ \ \ \ \ \ \ \ \ \ \ \ \ \ \ \ \ \ \ \ \ \ \ \ \ \ \ \ \ \ \ \ \ \ \ \ \ \ \ \ \ \ \ \ \ \ \ \ \ \ \ \ \ \ \ \ \ \ \ \ \ \ \ \ \ \ \ \ \ \textbar{}\ \textbar{}}git
rm
имена\_файлов\texttt{\textbar{}\ удалить\ файл\ и/или\ каталог\ из\ индекса\ репозитория\ (при\ этом\ файл\ и/или\ каталог\ остаётся\ в\ локальной\ директории)\textbar{}\ \ \ \ \ \ \ \ \ \ \ \ \ \ \ \ \ \ \ \ \ \ \ \ \ \ \ \ \ \ \ \ \ \ \ \ \ \ \ \ \ \ \ \ \ \ \ \ \ \ \ \ \ \ \ \ \ \ \ \ \ \ \ \ \ \ \ \ \ \ \ \ \ \ \ \ \ \ \ \ \ \ \ \ \ \ \ \ \ \ \ \ \ \ \ \ \ \ \ \ \ \ \ \ \ \ \ \ \textbar{}\ \textbar{}}git
commit -am \enquote*{Описание
коммита}\texttt{\textbar{}\ сохранить\ все\ добавленные\ изменения\ и\ все\ изменённые\ файлы\textbar{}\ \textbar{}}git
checkout -b
имя\_ветки\texttt{\textbar{}\ создание\ новой\ ветки,\ базирующейся\ на\ текущей\textbar{}\ \textbar{}}git
checkout
имя\_ветки\texttt{\textbar{}\ переключение\ на\ некоторую\ ветку\ (при\ переключении\ на\ ветку,\ которой\ ещё\ нет\ в\ локальном\ репозитории,\ она\ будет\ создана\ и\ связана\ с\ удалённой)\textbar{}\ \textbar{}}git
push origin
имя\_ветки\texttt{\textbar{}\ отправка\ изменений\ конкретной\ ветки\ в\ центральный\ репозиторий\textbar{}\ \textbar{}}git
merge --no-ff
имя\_ветки\texttt{\textbar{}\ слияние\ ветки\ с\ текущим\ деревом\textbar{}\ \textbar{}}git
branch -d
имя\_ветки\texttt{\textbar{}\ удаление\ локальной\ уже\ слитой\ с\ основным\ деревом\ ветки\textbar{}\ \textbar{}}git
branch -d
имя\_ветки\texttt{\textbar{}\ принудительное\ удаление\ локальной\ ветки\textbar{}\ \textbar{}}git
push origin :имя\_ветки` & удаление ветки с центрального репозитория \\
\end{longtable}

\chapter{Выполнение лабораторной
работы}\label{ux432ux44bux43fux43eux43bux43dux435ux43dux438ux435-ux43bux430ux431ux43eux440ux430ux442ux43eux440ux43dux43eux439-ux440ux430ux431ux43eux442ux44b}

\begin{enumerate}
\def\labelenumi{\arabic{enumi}.}
\tightlist
\item
  Настройка GitHub Перешла на сайт GitHub (https://github.com/) и
  создала там учетную запись (рис.1)
\end{enumerate}

\begin{figure}[H]

{\centering \pandocbounded{\includegraphics[keepaspectratio]{./Downloads/1.jng}}

}

\caption{рис.1}

\end{figure}%

\begin{enumerate}
\def\labelenumi{\arabic{enumi}.}
\setcounter{enumi}{1}
\tightlist
\item
  Базовая настройка git После создания запускаю виртуальную машину
  ubuntu. В терминале делаю предварительную конфигурацию git, вводя своё
  имя и email. (рис. 2.1)
\end{enumerate}

\begin{figure}[H]

{\centering \pandocbounded{\includegraphics[keepaspectratio]{./Downloads/2.1.jng}}

}

\caption{рис.2.1}

\end{figure}%

Потом я настроила utf-8 в выводе сообщений git для правильного
отображения символов (рис.2.2)

\begin{figure}[H]

{\centering \pandocbounded{\includegraphics[keepaspectratio]{./Downloads/2.2.jng}}

}

\caption{рис.2.2}

\end{figure}%

Задаю название «master» для начальной ветки. (рис. 2.3)

\begin{figure}[H]

{\centering \pandocbounded{\includegraphics[keepaspectratio]{./Downloads/2.3.jng}}

}

\caption{рис.2.3}

\end{figure}%

Ввожу параметры autocrlf и safecrlf с помощью команд ниже. (рис.2.4)

\begin{figure}[H]

{\centering \pandocbounded{\includegraphics[keepaspectratio]{./Downloads/2.4.jng}}

}

\caption{рис.2.4}

\end{figure}%

\begin{enumerate}
\def\labelenumi{\arabic{enumi}.}
\setcounter{enumi}{2}
\tightlist
\item
  Создание ssh-ключа Первым делом генерирую несколько приватных и
  открытых ключей. Для этого применяю команду ssh-keygen -C
  \enquote{aasidorova\href{mailto:1032256488@pfur.ru}{\nolinkurl{1032256488@pfur.ru}}}.
  Ключи сохраняются в каталоге \textasciitilde/.ssh/. (рис.3.1)
\end{enumerate}

\begin{figure}[H]

{\centering \pandocbounded{\includegraphics[keepaspectratio]{./Downloads/3.1.jng}}

}

\caption{рис.3.1}

\end{figure}%

Скопировала ключ в буфер обмена (рис. 3.2).

\begin{figure}[H]

{\centering \pandocbounded{\includegraphics[keepaspectratio]{./Downloads/3.2.jng}}

}

\caption{рис.3.2}

\end{figure}%

Вставила этот код на сайте GitHub в разделе SSH and GPG keys. (рис. 3.3)

\begin{figure}[H]

{\centering \pandocbounded{\includegraphics[keepaspectratio]{./Downloads/3.3.jng}}

}

\caption{рис.3.3}

\end{figure}%

\begin{enumerate}
\def\labelenumi{\arabic{enumi}.}
\setcounter{enumi}{3}
\tightlist
\item
  Создание рабочего пространства и репозитория курса на основе шаблона
  Создаю каталог для предмета для предмета \enquote{Архитектура
  компьютера} (рис.4).
\end{enumerate}

\begin{figure}[H]

{\centering \pandocbounded{\includegraphics[keepaspectratio]{./Downloads/4.jng}}

}

\caption{рис.4}

\end{figure}%

\begin{enumerate}
\def\labelenumi{\arabic{enumi}.}
\setcounter{enumi}{4}
\tightlist
\item
  Создание репозитория курса на основе шаблона Перехожу по ссылке
  https://github.com/yamadharma/course-directory-student template и
  выбираю Use this template(рис.5.1) В открывшимся окне указываю
  название и создаю репозиторий. (рис.5.2)
\end{enumerate}

\pandocbounded{\includegraphics[keepaspectratio]{./Downloads/5.1.jng}}
\pandocbounded{\includegraphics[keepaspectratio]{./Downloads/5.2.jng}}

Затем открываю терминал и перехожу в каталог курса с помощью cd
\textasciitilde/work/study/2025--2026/\enquote{Архитектура компьютера}
(рис.5.3)

\begin{figure}[H]

{\centering \pandocbounded{\includegraphics[keepaspectratio]{./Downloads/5.3.jng}}

}

\caption{рис.5.3}

\end{figure}%

Клонирую созданный репозиторий (рис. 5.4), используя ссылку,
скопированную с сайта. (рис. 5.5)

\pandocbounded{\includegraphics[keepaspectratio]{./Downloads/5.4.jng}}
\pandocbounded{\includegraphics[keepaspectratio]{./Downloads/5.5.jng}}

\begin{enumerate}
\def\labelenumi{\arabic{enumi}.}
\setcounter{enumi}{5}
\tightlist
\item
  Настройка каталога курса Я перешла в каталог курса: cd
  \textasciitilde/work/study/2023-2024/\enquote{Архитектура
  компьютера}/arch-p и создала необходимые каталоги: echo arch-pc
  \textgreater{} COURSE make prepare
\end{enumerate}

\pandocbounded{\includegraphics[keepaspectratio]{./Downloads/6.1.jng}}
\pandocbounded{\includegraphics[keepaspectratio]{./Downloads/6.2.jng}}

Отправляю созданные каталоги с локального репозитория на сервер с
помощью git add, комментирую и сохраняю изменения как добавление курса с
помощью git commit.(рис.6.3)

\begin{figure}[H]

{\centering \pandocbounded{\includegraphics[keepaspectratio]{./Downloads/6.3.jng}}

}

\caption{рис.6.3}

\end{figure}%

Отправляю файлы на сервер (рис. 6.4).

\begin{figure}[H]

{\centering \pandocbounded{\includegraphics[keepaspectratio]{./Downloads/6.4.jng}}

}

\caption{рис.6.4}

\end{figure}%

Проверяю правильность создания иерархии рабочего пространства в
локальном репозитории и на странице github. (рис.6.5,6.6)

\pandocbounded{\includegraphics[keepaspectratio]{./Downloads/6.5.jng}}
\pandocbounded{\includegraphics[keepaspectratio]{./Downloads/6.6.jng}}

\begin{enumerate}
\def\labelenumi{\arabic{enumi}.}
\setcounter{enumi}{6}
\tightlist
\item
  Задание для самостоятельной работы С помощью утилиты cd перехожу в
  labs/lab02/report (рис. 7.1) и создаю файл для отчета по этой
  лабораторной работе. (рис. 7.2)
\end{enumerate}

\pandocbounded{\includegraphics[keepaspectratio]{./Downloads/7.1.jng}}
\pandocbounded{\includegraphics[keepaspectratio]{./Downloads/7.2.jng}}

Перемещаюсь в каталог предыдущей лабораторной работы (рис.7.3)

\begin{figure}[H]

{\centering \pandocbounded{\includegraphics[keepaspectratio]{./Downloads/7.3.jng}}

}

\caption{рис.7.3}

\end{figure}%

Проверяю расположение файла с отчетом по лабораторной работе №1.
(рис.7.4)

\begin{figure}[H]

{\centering \pandocbounded{\includegraphics[keepaspectratio]{./Downloads/7.4.jng}}

}

\caption{рис.7.4}

\end{figure}%

Перемещаю этот файл в каталог lab01/report с помощью команды mv,
проверяю наличие файла. (рис.7.5)

\begin{figure}[H]

{\centering \pandocbounded{\includegraphics[keepaspectratio]{./Downloads/7.5.jng}}

}

\caption{рис.7.5}

\end{figure}%

С помощью команды git add добавила в коммит отчёты по двум
лабораторнымработам. Отправляю в центральный репозиторий сохраненные
изменения. (рис 7.6, рис.7.7).

\pandocbounded{\includegraphics[keepaspectratio]{./Downloads/7.6.jng}}
\pandocbounded{\includegraphics[keepaspectratio]{./Downloads/7.7.jng}}

Проверяю правильность выполненных действий на сайте. На рисунке 7.8 я
убеждаюсь в том, что указанный мной комментарий отображается (Add
existing file), и на рисунке 7.9 отображается добавленный файл.

\pandocbounded{\includegraphics[keepaspectratio]{./Downloads/7.8.jng}}
\pandocbounded{\includegraphics[keepaspectratio]{./Downloads/7.9.jng}}

После создания отчета по этой лабораторной работе я делаю действия
аналогичные, показанным в рисунках 7.4-7.9

\chapter{Вывод}\label{ux432ux44bux432ux43eux434}

В ходе работы я приобрела практические навыки работы с системой контроля
версий GitHub, разобрав следующие команды: clone, add, commit, push.
Также я изучила идеологию и применение подобных систем.


\printbibliography



\end{document}
